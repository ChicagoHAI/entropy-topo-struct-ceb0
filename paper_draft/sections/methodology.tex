\section{Methodology}

Our analysis proceeds through a systematic proof strategy that establishes theoretical bounds on the probability of generating novel mathematical structures while maintaining rigorous quality constraints. We develop this framework through several key steps:

First, we establish fundamental entropy bounds on the solution space $\mathcal{S}$ using Lemma 1, which gives:

\[H(\mathcal{S}) \geq \alpha \log(|\mathcal{B}|) + \beta D(\mathcal{S})\]

where $H(\mathcal{S})$ represents the entropy of the solution space, $|\mathcal{B}|$ is the size of the basis set, and $D(\mathcal{S})$ measures topological complexity.

We then apply Lemma 2 to restrict our analysis to the high-quality subspace $\mathcal{S}_Q \subseteq \mathcal{S}$ defined by:

\[\mathcal{S}_Q = \{s \in \mathcal{S} : Q(s) \geq q_{\text{min}}\}\]

where $Q(s)$ represents our quality metric and $q_{\text{min}}$ is our threshold parameter.

The probability of generating structures at distance $d$ from the basis is bounded using kernel methods from Lemma 3:

\[P(d|\mathcal{S}_Q) \leq C_1 \exp(-\lambda d) + C_2 \frac{V(d)}{|\mathcal{S}_Q|}\]

where $V(d)$ represents the volume of the $d$-sphere in our metric space.

Topological bounds from Lemma 4 establish the density of novel structures:

\[\rho_{\text{novel}}(\mathcal{S}_Q) \geq \gamma \frac{\log(|\mathcal{S}_Q|)}{D(\mathcal{S}_Q)}\]

We validate our theoretical framework through:

1. Numerical simulations implementing the structure generation process using established creativity frameworks, specifically:
   \[P_{\text{gen}}(s) = \frac{1}{Z} \exp(-\beta E(s))\]
   where $E(s)$ represents our energy function incorporating both novelty and quality metrics.

2. Empirical probability calculations across multiple mathematical domains including graph theory, number theory, and algebraic structures, measuring:
   \[P_{\text{emp}}(d) = \frac{N(d)}{N_{\text{total}}}\]

3. Statistical comparison between theoretical bounds and empirical results using the Kolmogorov-Smirnov test with significance level $\alpha = 0.05$.

4. Validation against historical mathematical discovery data, comparing predicted and observed novelty distributions using:
   \[\chi^2 = \sum_{i=1}^{k} \frac{(O_i - E_i)^2}{E_i}\]

This methodology enables us to establish rigorous bounds on the probability of generating novel mathematical structures while maintaining specified quality constraints.