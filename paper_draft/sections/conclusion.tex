\section{Conclusion}

In this work, we established fundamental theoretical bounds on the probability of generating novel mathematical structures within a solution space, characterized by its topological invariant and creativity entropy. Our main contribution provides a rigorous upper bound:

\[P(\mathcal{N}) \leq \exp(-\alpha H(S)) + \beta\tau(S)\]

This result significantly extends previous work on mathematical creativity by providing quantifiable metrics for assessing the likelihood of novel mathematical discovery. While \cite{varshney2019mathematical} introduced preliminary frameworks for analyzing computational creativity in mathematics, our bounds explicitly incorporate both the topological structure and entropy of the solution space.

The proven relationship between creativity entropy $H(S)$ and the basis set of fundamental structures provides a practical tool for evaluating the potential for novel mathematical discovery in different domains. This advances beyond the empirical approaches discussed in \cite{fuat2025exploring} by offering a theoretical foundation for predicting where novel mathematical structures are most likely to emerge.

Several important questions remain open for future investigation:
\begin{itemize}
\item Determining optimal values for the domain-specific constants $\alpha$ and $\beta$
\item Extending the bounds to infinite-dimensional solution spaces
\item Developing computational methods for approximating $\tau(S)$ in complex mathematical domains
\end{itemize}

Our results suggest that the probability of generating novel mathematical structures decreases exponentially with increasing creativity entropy, while maintaining a linear dependence on topological invariants. This provides theoretical guidance for directing mathematical research toward promising areas with favorable entropy-topology combinations.