\section{Introduction}

The systematic generation of novel mathematical concepts and structures remains one of the most challenging problems in mathematical research and computational creativity. While human mathematicians have historically relied on intuition and insight to discover new mathematical objects and relationships, there is growing interest in developing rigorous frameworks for understanding and potentially automating this creative process \cite{varshney2019mathematical}.

Recent work has demonstrated that computational approaches can support mathematical creativity across various domains, from educational settings \cite{eli2025parental} to advanced research. However, fundamental questions remain about the theoretical limits of generating novel mathematical structures that satisfy rigorous quality criteria. As highlighted by \cite{her2024investigating}, the emergence of generative AI systems has brought these questions to the forefront of mathematical research.

The challenge lies in characterizing the relationship between a solution space's inherent properties and the probability of discovering meaningful new mathematical structures within it. While previous studies have explored creativity in mathematical problem-solving \cite{usodo2020relational}, a formal theoretical framework for analyzing the generation of novel mathematical concepts has been lacking.

In this paper, we present a fundamental bound on the probability of generating novel mathematical structures within a given solution space. Our main result establishes that for any mathematical solution space $S$ with topological invariant $\tau(S)$ and creativity entropy $H(S)$, the probability $P(\mathcal{N})$ of generating a novel mathematical structure $\mathcal{N}$ satisfying quality threshold $q$ is bounded by:

\[
P(\mathcal{N}) \leq \exp(-\alpha H(S)) + \beta\tau(S)
\]

where $\alpha, \beta > 0$ are domain-specific constants. This bound provides fundamental insights into the relationship between a solution space's complexity (measured by its creativity entropy) and its topological properties in determining the feasibility of discovering novel mathematical structures.

Our contributions include:
\begin{itemize}
\item A formal definition of creativity entropy for mathematical solution spaces
\item A novel theoretical framework linking topological invariants to creative potential
\item Rigorous proof of the probability bound for novel structure generation
\item Applications to specific mathematical domains with explicit calculations of the bounds
\end{itemize}

The remainder of this paper is organized as follows. Section 2 presents the theoretical framework and definitions. Section 3 proves the main probability bound. Section 4 discusses applications and implications, and Section 5 concludes with directions for future research.