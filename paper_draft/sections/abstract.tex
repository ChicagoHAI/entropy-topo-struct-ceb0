\begin{abstract}
We present a theoretical framework for analyzing the probability of generating novel mathematical structures within a given solution space. Our main result establishes fundamental bounds on the likelihood of discovering mathematically novel concepts that satisfy specified quality criteria. Specifically, for a solution space $S$ with topological invariant $\tau(S)$ and creativity entropy $H(S)$, we prove that the probability $P(\mathcal{N})$ of generating a novel mathematical structure $\mathcal{N}$ meeting quality threshold $q$ is bounded by $P(\mathcal{N}) \leq \exp(-\alpha H(S)) + \beta\tau(S)$ for domain-specific constants $\alpha, \beta > 0$. The proof combines techniques from information theory, algebraic topology, and statistical learning theory. We first establish entropy lower bounds on the solution space, then restrict our analysis to a high-quality subspace using measure-theoretic arguments. Through careful application of kernel methods and topological invariants, we characterize the density of novel structures within the restricted space. Our results provide theoretical insights into the fundamental limits of automated mathematical discovery and suggest practical guidelines for designing algorithms to generate novel mathematical concepts. These bounds have implications for computer-assisted mathematics and the development of automated theorem provers.
\end{abstract}